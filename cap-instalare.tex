%% LyX 2.0.6 created this file.  For more info, see http://www.lyx.org/.
%% Do not edit unless you really know what you are doing.
\documentclass[11pt,romanian]{book}
\usepackage[utf8x]{inputenc}
\setcounter{secnumdepth}{3}
\setcounter{tocdepth}{3}

\makeatletter

%%%%%%%%%%%%%%%%%%%%%%%%%%%%%% LyX specific LaTeX commands.
\newcommand{\noun}[1]{\textsc{#1}}

\makeatother

\usepackage{babel}
\begin{document}

\section*{Cerinţe }

Pentru a instala MOODLE este nevoie: 
\begin{enumerate}
\item Un Web-server (în majoritatea cazurilor Apache deşi poate fi utilizat
şi IIS de la Micrsoft) 
\item Interpretatorul PHP ( versiunea nu mai jos de 4.1.0). MOODLE de la
versiunea 1.4 menţine versiunea 5
\item Un server de baze de date MySQL sau PostgreSQL.
\end{enumerate}
Companiile de hosting de regulă implicit satisfac aceste condiţii.
În caz contrar aflaţi modalitatea de a primi aceste ori schimbaţi
compania care oferă hosting. 


\section*{Descărcarea, copierea şi amplasarea fişierelor }

Există două posibilităţi de descărcare a MOODLE, ca un fişier arhivat
sau prin sistemul CVS de pe pagina de descărcare www.moodle.org. După
descărcare şi dezarhivare sau verificare prin CVS, pe calculator va
apărea o mapa moodle. Ea poate fi amplasată în mapa documentelor a
serverului web. Dacă descărcaţi MOODLE pe calculatorul local şi apoi
îl plasţi pe server mai optimal este de al plasa ca un fişier arhivat
şi dezarhivarea să fie făcută în direct pe server. Companie de hosting
oferă interfaţe de tipul CPanel sau DirectAdmin care au aşa posibilităţi.


\section*{Lansarea scriptului de instalare}

Pentru a lansa scriptul de instalare (index.php) e nevoe de culege
în browser adresa sit-ului sau direct http://serverul/index.php. Scenariu
de lansare cere să fie salvate cookie. E nevoe să confirmaţi salvarea.

Scenariu nu face alt ceva decît generează fişierul config.php pentru
server.

Paralel se va verfica şi configurările serverului. De regulă pot apîrea
probleme cu unile extensiuni PHP.


\subsubsection{Configurările serverului Web}

În primul rînd asiguraţi-vă că serverul e configurat pentru a folosi
implicit pagina index.php (eventual ca compliment la index.html, index.htm,
default.htm etc.)

În Apache acest lucru se configurează, utilizînd paramentrul \noun{DirectoryIndex
}în fişierul httpd.conf. 

Exemplu:

\noun{DirectoryIndex index.php index.html, ieendex.htm, defoult.htm}

Asiguraţi-vă ca index.php este în listă şi e de dorit primul (pentru
efecienţă)

Dacă utilizaţi Apache2 e bine de schimba opţiunea \noun{AcceptPathInfo}
care face posibilă lucrul scenariilor de transmitere a argumentelor
\noun{http://server/fişier.php/argument1/argument2.} 
\end{document}
